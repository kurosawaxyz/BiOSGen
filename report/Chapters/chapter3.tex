\chapter{Commandes utiles}
\section{Quelques commandes}

Voici quelques commandes utiles : \cite{latex:companion}

%------ Pour insérer et citer une image centralisée -----
\subsection{Commande pour insérer et citer une image centralisée}

\begin{figure}[h!]
    \centering
    \includegraphics[width=0.5\linewidth]{logos/ensias.png}
    \caption{Légende de la figure}
    \label{fig:Label de la figure}
\end{figure}
% Le premier argument est le chemin pour la photo
% Le deuxième est la hauteur de la photo
% Le troisième la légende
% Le quatrième le label
Ici, je cite l'image \ref{fig:Label de la figure}


%------- Pour insérer et citer une équation --------------
\subsection{Commande pour insérer et citer une équation}
\begin{equation} \label{eq: exemple}
\rho + \Delta = 42
\end{equation}

L'équation \ref{eq: exemple} est cité ici. \cite{knuth:1984}

% ------- Pour écrire des variables ----------------------

Pour écrire des variables dans le texte, il suffit de mettre le symbole \$ entre le texte souhaité comme : constante $\rho$. 

%------- Pour ecrire un code ------------------------------
\newpage
\subsection{Commande pour ecrire un code}

%linenos pour avoir les nombres des lignes 
%specifier la language de programmation utilisé

\begin{listing}[!ht]
\begin{minted}[bgcolor=bg,linenos]{C}
#ifndef DECLARATION_H_INCLUDED
#define DECLARATION_H_INCLUDED

//declaration de structure qui englobe les donnees d'un adherant
typedef struct _donnee
{
    char code[20];
    char nom[20];
    char domaineInteret[20];
}donnees;

#endif // DECLARATION_H_INCLUDED
\end{minted}
\caption{exemple de caption de code C}
\label{listing 1}
\end{listing}

One-line code formatting also works with \texttt{minted}. For example, a small fragment of HTML like this:
\mint{html}|<h2>Something <b>here</b></h2>|
\noindent can be formatted correctly.



\LaTeX{} \cite{latex2e} is a set of macros built atop \TeX{} \cite{texbook}.
